\documentclass[11pt]{article}

\input{../../Latex_Common/skinnerr_latex_preamble_asen5417.tex}

%%
%% DOCUMENT START
%%

\begin{document}

\pagestyle{fancyplain}
\lhead{}
\chead{}
\rhead{}
\lfoot{\hrule ASEN 5417: Homework 8}
\cfoot{\hrule \thepage}
\rfoot{\hrule Ryan Skinner}

\noindent
{\Large Homework 8}
\hfill
{\large Ryan Skinner}
\\[0.5ex]
{\large ASEN 5417: Numerical Methods}
\hfill
{\large Due 2015/12/8}\\
\hrule
\vspace{6pt}

%%%%%%%%%%%%%%%%%%%%%%%%%%%%%%%%%%%%%%%%%%%%%%%%%
%%%%%%%%%%%%%%%%%%%%%%%%%%%%%%%%%%%%%%%%%%%%%%%%%
\section{Introduction} %%%%%%%%%%%%%%%%%%%%%%%%%%
%%%%%%%%%%%%%%%%%%%%%%%%%%%%%%%%%%%%%%%%%%%%%%%%%
%%%%%%%%%%%%%%%%%%%%%%%%%%%%%%%%%%%%%%%%%%%%%%%%%

Consider the linear convection-diffusion equation
\begin{equation}
\pp{T}{t} + v \pp{T}{y} = \frac{1}{\text{Pr}\;\text{Re}} \pp{^2T}{y^2}
\;, \qquad
v = \sin(\pi y)
\;,
\label{eq:conv-diff}
\end{equation}
subject to the initial conditions
\begin{equation}
\begin{aligned}
\text{(a)} && T(y,t=0) &= \cos(2 \pi y) \sin(\pi y) \\
\text{(b)} && T(y,t=0) &= \cos(2 \pi y)
\;,
\end{aligned}
\end{equation}
and parameters
\begin{equation}
\begin{aligned}
\text{Re} &= 1 &\quad &\text{(Reynolds number, molten glass)} \\
\text{Pr} &= 25 &\quad &\text{(Prandtl number, molten glass)} \\
\Delta t &= 0.001 &\quad &\text{(Time step)} \\
L_y &= 2 &\quad &\text{(Domain $y$-length)} \\
N &= 2^n + 1 &\quad &\text{(Number of $y$-points, where $n=5,6$)}
\;.
\end{aligned}
\end{equation}

\subsection{Problem 1}

Use the Fourier pseudo-spectral method to numerically integrate \eqref{eq:conv-diff} with the given parameters. Use the Euler explicit method for time advancement. Higher resolution with $n=6$ will improve the accuracy of the method for the initial condition (a). Plot $T$ as a function of time $t$ at $t = \{0.2, 2, 5, 10\}$.

\subsection{Problem 2}

Use the FTCS Euler explicit method with second-order finite differences for the same computation, and compare results to the Fourier pseudo-spectral method using the same mesh resolution.

%%%%%%%%%%%%%%%%%%%%%%%%%%%%%%%%%%%%%%%%%%%%%%%%%
%%%%%%%%%%%%%%%%%%%%%%%%%%%%%%%%%%%%%%%%%%%%%%%%%
\section{Methodology} %%%%%%%%%%%%%%%%%%%%%%%%%%%
%%%%%%%%%%%%%%%%%%%%%%%%%%%%%%%%%%%%%%%%%%%%%%%%%
%%%%%%%%%%%%%%%%%%%%%%%%%%%%%%%%%%%%%%%%%%%%%%%%%

\subsection{Problem 1}

We re-write \eqref{eq:conv-diff} as
\begin{equation}
\pp{T}{t} = \frac{1}{\text{Pr}\;\text{Re}} \pp{^2T}{y^2} - v \pp{T}{y} 
\;, \qquad
v = \sin(\pi y)
\;.
\end{equation}
The first and second spatial derivatives of $T$ are calculated by taking the Fourier transform of $T$, multiplying the Fourier coefficients by $i k_n$ and $-k_n^2$, respectively, and then taking the inverse Fourier transform. With values of $\partial T / \partial t$ known at all grid points now, values of $T$ at the next time step are calculated using the explicit Euler method,
\begin{equation}
T^{n+1} = T^n + \pp{T}{t}
\;.
\end{equation}

%%%%%%%%%%%%%%%%%%%%%%%%%%%%%%%%%%%%%%%%%%%%%%%%%
%%%%%%%%%%%%%%%%%%%%%%%%%%%%%%%%%%%%%%%%%%%%%%%%%
\section{Results} %%%%%%%%%%%%%%%%%%%%%%%%%%%%%%%
%%%%%%%%%%%%%%%%%%%%%%%%%%%%%%%%%%%%%%%%%%%%%%%%%
%%%%%%%%%%%%%%%%%%%%%%%%%%%%%%%%%%%%%%%%%%%%%%%%%

%\begin{figure}[p!]
%\begin{center}
%\includegraphics[width=0.85\textwidth]{Prob1_C075.eps} \\
%\includegraphics[width=0.85\textwidth]{Prob1_C100.eps} \\
%\includegraphics[width=0.85\textwidth]{Prob1_C125.eps} \\
%\includegraphics[width=0.85\textwidth]{Prob1_t8.eps}
%\\[0.5cm]
%\caption{MacCormack method solution for different Courant numbers, and comparison at time $t \sim 8$. The Courant number is seen to substantially affect the accuracy of the results.}
%\label{fig:MacCormack}
%\end{center}
%\end{figure}

%%%%%%%%%%%%%%%%%%%%%%%%%%%%%%%%%%%%%%%%%%%%%%%%%
%%%%%%%%%%%%%%%%%%%%%%%%%%%%%%%%%%%%%%%%%%%%%%%%%
\section{Discussion} %%%%%%%%%%%%%%%%%%%%%%%%%%%%
%%%%%%%%%%%%%%%%%%%%%%%%%%%%%%%%%%%%%%%%%%%%%%%%%
%%%%%%%%%%%%%%%%%%%%%%%%%%%%%%%%%%%%%%%%%%%%%%%%%


%%%%%%%%%%%%%%%%%%%%%%%%%%%%%%%%%%%%%%%%%%%%%%%%%
%%%%%%%%%%%%%%%%%%%%%%%%%%%%%%%%%%%%%%%%%%%%%%%%%
\section{References} %%%%%%%%%%%%%%%%%%%%%%%%%%%%
%%%%%%%%%%%%%%%%%%%%%%%%%%%%%%%%%%%%%%%%%%%%%%%%%
%%%%%%%%%%%%%%%%%%%%%%%%%%%%%%%%%%%%%%%%%%%%%%%%%

No external references were used other than the course notes for this assignment.

%%%%%%%%%%%%%%%%%%%%%%%%%%%%%%%%%%%%%%%%%%%%%%%%%
%%%%%%%%%%%%%%%%%%%%%%%%%%%%%%%%%%%%%%%%%%%%%%%%%
\section*{Appendix: MATLAB Code} %%%%%%%%%%%%%%%%
%%%%%%%%%%%%%%%%%%%%%%%%%%%%%%%%%%%%%%%%%%%%%%%%%
%%%%%%%%%%%%%%%%%%%%%%%%%%%%%%%%%%%%%%%%%%%%%%%%%

The following code listings generate all figures presented in this homework assignment.

%\includecode{Problem_1.m}

%%
%% DOCUMENT END
%%
\end{document}
